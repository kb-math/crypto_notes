\documentclass[twoside, a4paper, 10pt]{amsart}
\title[ ]{Notes on Cryptography}
%\usepackage{amsaddr}
%\email{Kamil.Bulinski@minetec.com.au}
\usepackage{amsfonts}
\usepackage{amsthm}
\usepackage{verbatim}
\usepackage{amsmath, amssymb}
\usepackage{tikz}
\usetikzlibrary{matrix, arrows}
\usepackage{listings}
\usepackage{color}
\usepackage{listings}
\usepackage[all]{xy}
\usepackage[pdftex,colorlinks,linkcolor=blue,citecolor=blue]{hyperref}
\usepackage{graphicx}
\usepackage{float}
\usepackage[margin=3cm]{geometry}
\usepackage{bigints}
\usepackage{dsfont}
\setlength{\textwidth}{6.5in}
\setlength{\oddsidemargin}{0in}
\setlength{\evensidemargin}{0in}
\setlength{\parindent}{0pt}
\setlength{\parskip}{1ex plus 0.5ex minus 0.2ex}
\linespread{1.3}


\setcounter{secnumdepth}{4}

\begin{document}
\maketitle
\raggedbottom


%% Mathcal large
\newcommand{\cA}{\mathcal{A}}
\newcommand{\cB}{\mathcal{B}}
\newcommand{\cC}{\mathcal{C}}
\newcommand{\cD}{\mathcal{D}}
\newcommand{\cE}{\mathcal{E}}
\newcommand{\cF}{\mathcal{F}}
\newcommand{\cG}{\mathcal{G}}
\newcommand{\cH}{\mathcal{H}}
\newcommand{\cI}{\mathcal{I}}
\newcommand{\cJ}{\mathcal{J}}
\newcommand{\cK}{\mathcal{K}}
\newcommand{\cL}{\mathcal{L}}
\newcommand{\cM}{\mathcal{M}}
\newcommand{\cN}{\mathcal{N}}
\newcommand{\cO}{\mathcal{O}}
\newcommand{\cP}{\mathcal{P}}
\newcommand{\cQ}{\mathcal{Q}}
\newcommand{\cR}{\mathcal{R}}
\newcommand{\cS}{\mathcal{S}}
\newcommand{\cT}{\mathcal{T}}
\newcommand{\cU}{\mathcal{U}}
\newcommand{\cV}{\mathcal{V}}
\newcommand{\cW}{\mathcal{W}}
\newcommand{\cX}{\mathcal{X}}
\newcommand{\cY}{\mathcal{Y}}
\newcommand{\cZ}{\mathcal{Z}}
%% Mathbb large
\newcommand{\bA}{\mathbb{A}}
\newcommand{\bB}{\mathbb{B}}
\newcommand{\bC}{\mathbb{C}}
\newcommand{\bD}{\mathbb{D}}
\newcommand{\bE}{\mathbb{E}}
\newcommand{\bF}{\mathbb{F}}
\newcommand{\bG}{\mathbb{G}}
\newcommand{\bH}{\mathbb{H}}
\newcommand{\bI}{\mathbb{I}}
\newcommand{\bJ}{\mathbb{J}}
\newcommand{\bK}{\mathbb{K}}
\newcommand{\bL}{\mathbb{L}}
\newcommand{\bM}{\mathbb{M}}
\newcommand{\bN}{\mathbb{N}}
\newcommand{\bO}{\mathbb{O}}
\newcommand{\bP}{\mathbb{P}}
\newcommand{\bQ}{\mathbb{Q}}
\newcommand{\bR}{\mathbb{R}}
\newcommand{\bS}{\mathbb{S}}
\newcommand{\bT}{\mathbb{T}}
\newcommand{\bU}{\mathbb{U}}
\newcommand{\bV}{\mathbb{V}}
\newcommand{\bW}{\mathbb{W}}
\newcommand{\bX}{\mathbb{X}}
\newcommand{\bY}{\mathbb{Y}}
\newcommand{\bZ}{\mathbb{Z}}


\newcounter{dummy} \numberwithin{dummy}{section}

\theoremstyle{definition}
\newtheorem{mydef}[dummy]{Definition}
\newtheorem{prop}[dummy]{Proposition}
\newtheorem{corol}[dummy]{Corollary}
\newtheorem{thm}[dummy]{Theorem}
\newtheorem{lemma}[dummy]{Lemma}
\newtheorem{eg}[dummy]{Example}
\newtheorem{notation}[dummy]{Notation}
\newtheorem{remark}[dummy]{Remark}
\newtheorem{claim}[dummy]{Claim}
\newtheorem{Exercise}[dummy]{Exercise}
\newtheorem{question}[dummy]{Question}
\newtheorem{conjecture}[dummy]{Conjecture}


Based on Katz-Lindell. All random variables considered discrete?

\section{Introduction and perfect secrecy}

\textbf{Encryption Scheme:} An encryption scheme consists of a tuple $(\operatorname{Gen}, \operatorname{Enc}, \operatorname{Dec})$ where 

\begin{itemize} 
	\item $\operatorname{Gen}$ is a random variable with values in a set $\mathcal{K}$, called the \textit{keyspace}. 
	\item There is a set $\mathcal{M}$, called the \textit{message space}. 
	\item There is a set $\mathcal{C}$, each element is called a \textit{ciphertext}. 
	\item For each $k \in \mathcal{K}$ and $m \in \mathcal{M}$ we have a random variable $\operatorname{Enc}_k(m)$ taking values in $\mathcal{C}$.
	\item For each $k \in \mathcal{K}$, we have a map $\operatorname{Dec}_k:\mathcal{C} \to \mathcal{M}$ satisfying $$\operatorname{Dec}_k(\operatorname{Enc}_k(m)) = m \quad \text{ for all } m  \in \mathcal{M}$$. 

\end{itemize}

More formally, $\operatorname{Enc}_k(m): \Omega \to \mathcal{C}$ on some sample space $\Omega$. The last point means that $\operatorname{Dec}_k(\operatorname{Enc}_k(m)(\omega)) = m$ for all $\omega \in \Omega$. If $\operatorname{Enc}_k(m)$ is a constant map, we simply consider it to be a deterministic function $\operatorname{Enc}_k: \mathcal{M} \to \mathcal{C}$.

\begin{eg}[Caeser cipher] In this example, we identify naturally the lowercase letters $\{a, b, \ldots, z\}$ with $\mathbb{Z}/26\bZ$. We let $\mathcal{K} = \mathbb{Z}/26\bZ$ and we let $\mathcal{M} = \mathcal{C}$ be the set of all words in the lowercase letters. We now define $\operatorname{Enc}_k(m) \in \mathcal{C}$ to be the word obtained by adding $k$ (mod 26) to each letter of $m$. So $\operatorname{Enc}_3(zac) = cdf $. So $\operatorname{Dec}_k = \operatorname{Enc}_{-k}$ is the inverse. The distribution on $\mathcal{K}$ is uniform (i.e., $\operatorname{Gen}$ takes values uniformly in $\mathcal{K}$).

\end{eg}

\begin{mydef}[Perfect secrecy] An encryption scheme is said to be \textit{perfectly secret} if for all $m,m' \in \mathcal{M}$ and $c \in \mathcal{C}$ we have that $$Pr(\operatorname{Enc}_k(m) = c) = Pr(\operatorname{Enc}_k(m') = c).$$ More precisely, if $P_{\mathcal{K}}$ is the distribution on $\mathcal{K}$ and $P_{\mathcal{C},m,k}$ is the distribution on $\mathcal{C}$ induced by $\operatorname{Enc}_k(m)$ then $$\sum_{k \in \mathcal{K}} P_{\mathcal{K}}(k) P_{\mathcal{C},k,m}(c) = \sum_{k \in \mathcal{K}} P_{\mathcal{K}}(k) P_{\mathcal{C},k,m'}(c).$$ That is, $Pr$ refers to the probability distribution where one chooses $k \in \mathcal{K}$ randomly (according to the random variable $\operatorname{Gen}$ part of the encryption scheme) and then one runs the random variable $\operatorname{Enc}_k(m)$ to produce $c \in \mathcal{C}$. If $\operatorname{Enc}_k$ is deterministic ($\operatorname{Enc}_k(m)$ is constant for all $k,m$) then obviously $Pr$ is just $P_{\mathcal{K}}$.

\end{mydef}

\begin{eg}[Caeser cipher is not perfectly secret] Let $m = aa$ and $m' = ab$ and let $c = ff$. Then $Pr(\operatorname{Enc}_k(m) = c) = Pr( k = 0) = \frac{1}{26}$, i.e., the probability that $aa$ gets encoded into $ff$ happens only if $k = 5$, so with probability $\frac{1}{26}$. However $Pr(\operatorname{Enc}_k(m') = c) = 0$ because $ab$ can only be encoded into one of $ab, bc, ce, \ldots, za$.

\end{eg}

\begin{eg}[one time pad] Let $G$ be a finite group. We define an encryption scheme where the keyspace and namespace is $G$. The encryption is $Enc_k(m) = km$. Decryption is given by $Dec_k(m) = k^{-1}m$. The distribution on keyspace is uniform. This scheme is perfectly secret as $$Pr(Enc_k(m) = c) = Pr(km = c) = Pr(k = cm^{-1}) = \frac{1}{|G|},$$ and this expression is clearly independent of $m$ for each fixed $c$. In the literature, the one-time pad is actually defined only for the case $G = (\mathbb{Z}/2\mathbb{Z})^{\ell}$. It is called the one-time pad because one needs to generate a new key for each message, i.e., if we send two differnet messages $m$ and $m'$ then the eavesdroper can compute $Enc_k(m) + Enc_k(m') = k + m + k+ m' = m+m'$. Thus the eavesdroper can compute the XOR of two different secret messages, which can be bad. Another drawback is the lack of efficiency in that the keyspace is as large as the message space.

\end{eg}

\begin{mydef} Consider an encryption scheme as above. Let $\mathcal{P}_{\mathcal{M}}$ be any distribution on the message space $\mathcal{M}$. We define the induced distribution $Pr$ on $\mathcal{K} \times \mathcal{M} \times \mathcal{C}$ to be the distribution where $$Pr(K = k, M=m, C=c) = \mathcal{P}_{\mathcal{K}}(k) \mathcal{P}_{\mathcal{M}}(m) P_{\mathcal{C},k,m}(c).$$ In other words we choose $k \in \mathcal{K}$ randomly and then $m \in \mathcal{M}$ independently  and then $c \in \mathcal{C}$ is drawn according to the random variable $\operatorname{Enc}_k(m)$.

\end{mydef}

\begin{prop} An encryption scheme is perfectly secret if and only if for any distribution $\mathcal{P}_{\mathcal{M}}$ the induced distribution $Pr$ satisfies the property that $$Pr(M=m ~|~ C=c) = Pr(M = m) \quad \text{ for all } m \in M \text{ and } c \in C \text{ with} Pr(C=c) >0. $$

\end{prop}

\begin{eg} Suppose we have a distribution where $\mathcal{P}_M(ab) = 0.4$ and $\mathcal{P}_M(aa) = 0.1$ and $\mathcal{P}_M(bb) = 0.5$. If we are using the Caeser cipher, then $Pr(M = ab `~|~ C = dd) = 0 \neq 0.4$ but clearly $P(C=dd) >0$. Thus the Caeser cipher is not perfectly secret.  

\end{eg}

\begin{mydef}[Adversary] \label{def: adversary} An \text{adversary} to a given encryption scheme $\Pi = (\operatorname{Gen}, \operatorname{Enc}, \operatorname{Dec})$ consists of, for each $m_0,m_1 \in \mathcal{M}$ and $c \in \mathcal{C}$, a random variable $\mathcal{A}(m_0, m_1, c)$ taking values in $\{0,1\}$. For each fixed $m_0,m_1$, we define two experiments, $\operatorname{Exp}_0$ and $\operatorname{Exp}_1$ as follows. For fixed $b \in \{0,1\}$ we define the random variable $\operatorname{Exp}_b$ by the following experiment:

\begin{enumerate}
	\item Choose $k \in \mathcal{K}$ randomly according to $\operatorname{Gen}$.
	\item Choose $c_b \in \mathcal{C}$ randomly according to $\operatorname{Enc}_k(m_b)$.
	\item Now send $c_b$ to the adversary. They then choose $b' \in \{0,1\}$ randomly according to $\mathcal{A}(m_0, m_1, c_b)$.
	\item Now define $\operatorname{Exp}_b \in \{0,1\}$ to be $1$ if $b = b'$ and $0$ if $b \neq b'$.
\end{enumerate}

 For each fixed $m_0,m_1$, we define a random variable $\operatorname{PrivK}_{\mathcal{A}, \Pi}$ as follows:

\begin{enumerate}
	\item Choose $b \in \{0,1\}$ uniformly randomly.
	\item Run $\operatorname{Exp}_b$ as above.
	\item Now define $\operatorname{PrivK}_{\mathcal{A}, \Pi} \in \{0,1\}$ to be the result of $\operatorname{Exp}_b$.

\end{enumerate}

\end{mydef}

\begin{prop} An encryption scheme $\Pi = (\operatorname{Gen}, \operatorname{Enc}, \operatorname{Dec})$ is perfectly secret if and only if for all adversaries $\mathcal{A}$, we have that $Pr(\operatorname{PrivK}_{\mathcal{A}, \Pi} = 1) = \frac{1}{2}. $

\end{prop}

Intuitively, the adversary constructs any $m_0,m_1$ that they wish, they then get an assistant to randomly choose $b \in \{0,1\}$ and pass $m_b$ into the encryption scheme and get a corresponding $c_b$. The adversary then has to guess, based on this triple $m_0, m_1, c_b$ a $b' \in \{0,1\}$ and try to get $b' = b$ (they do not know $b$, only the assistant does). If they can succeed (i.e., $(b == b')$) with probability greater than $\frac{1}{2}$, then the Proposition says that the scheme is not perfect.

To avoid confusing the different probabilitities, we let $$Pr_{\mathcal{A}}(\mathcal{A}(m_0,m_1, c) = b')$$ denote the probability that $\mathcal{A}(m_0,m_1,c) = b'$ where $m_0,m_1,c$ are \textbf{fixed} (so here randomness is purely dicated by the adversary $\mathcal{A}$ and not the experiment). While $$Pr_{\operatorname{Exp}_b} (\mathcal{A} = b')$$ denotes the probability that $\mathcal{A}$ returns $b'$ in $\operatorname{Exp}_b$.

\begin{proof}[Proof of Proposition] Note that $$ Pr(\operatorname{PrivK}_{\mathcal{A}, \Pi} = 1) = \frac{1}{2} Pr_{\operatorname{Exp}_0} (\mathcal{A} = 0) + \frac{1}{2} Pr_{\operatorname{Exp}_1}(\mathcal{A} = 1)$$

We compute the first term as $$\frac{1}{2} Pr_{\operatorname{Exp}_0} (\mathcal{A} = 0) = \frac{1}{2} \sum_{k \in \mathcal{K}} P_{\mathcal{K}}(k) \sum_{c \in \mathcal{C}}  P_{\mathcal{C}, k, m_0}(c) Pr_{\mathcal{A}}( \mathcal{A}(m_0, m_1,c) = 0 )$$ likewise, the second term is $$\frac{1}{2} Pr_{\operatorname{Exp}_1}(\mathcal{A} = 1) = \frac{1}{2} \sum_{k \in \mathcal{K}} P_{\mathcal{K}}(k) \sum_{c \in \mathcal{C}}  P_{\mathcal{C}, k, m_1}(c) Pr_{\mathcal{A}}( \mathcal{A}(m_0, m_1,c) = 1 )$$

now using the identity  $Pr_{\mathcal{A}}( \mathcal{A}(m_0, m_1,c) = 1 ) = 1- Pr_{\mathcal{A}}(\mathcal{A}(m_0, m_1, c) = 0)$ we can add these two terms to get 
$$ Pr(\operatorname{PrivK}_{\mathcal{A}, \Pi} = 1) = \frac{1}{2} \sum_{k \in \mathcal{K}} P_{\mathcal{K}}(k) \sum_{c \in \mathcal{C}}  P_{\mathcal{C}, k, m_1}(c)  + \frac{1}{2} \sum_{k \in \mathcal{K}} P_{\mathcal{K}}(k) \sum_{c \in \mathcal{C}}  Pr_{\mathcal{A}}(\mathcal{A}(m_0, m_1,c) = 0) (P_{\mathcal{C}, k, m_0}(c) - P_{\mathcal{C}, k, m_1}(c)).  $$ The first sum is  $$\frac{1}{2} \sum_{k \in \mathcal{K}} P_{\mathcal{K}}(k) \sum_{c \in \mathcal{C}}  P_{\mathcal{C}, k, m_1}(c) = \frac{1}{2}$$ as we are summing over the sample space of a probability measure. Now we have to show that the second sum vanishes for all $\mathcal{A}$ if and only if the scheme is perfectly secret. We rewrite this sum (ignoring the $\frac{1}{2}$ factor) as:

\begin{align*} & \sum_{c \in \mathcal{C}}Pr_{\mathcal{A}}(\mathcal{A}(m_0, m_1,c) = 0) \left( \sum_{k \in \mathcal{K}} P_{\mathcal{K}}(k) P_{\mathcal{C}, k, m_0}(c) -  \sum_{k \in \mathcal{K}} P_{\mathcal{K}}(k) P_{\mathcal{C}, k, m_1}(c) \right) \end{align*}

By definition of perfect secrecy, the term inside is $0$, i.e., because it is $$Pr(C = c | M = m_0) - Pr(C = c | M = m_1).$$ Conversely, suppose that this whole expression is $0$ for all $\mathcal{A}$. In particular, for each fixed $c_0 \in C$, if we choose $\mathcal{A}$ so that $Pr_{\mathcal{A}}(\mathcal{A}(m_0,m_1, c) = \mathds{1}_{\{c_0\}} (c)$ then we see that this expression is $$P(C = c_0 | M=m_0) - P(C = c_0 | M = m_1)$$ and is equal to $0$. \end{proof}

\begin{prop}\label{prop: perfect secrecy implies large keyspace} In a perfectly secret scheme, we have $|\mathcal{K}| \geq |\mathcal{M}|$.

\end{prop}

\begin{proof} Suppose for contradiction that $|\mathcal{K}| < |\mathcal{M}|$. Choose $c \in C$ such that $Pr(Enc_k(m) = c) > 0$ for some $k,m \in \mathcal{K} \times \mathcal{M}$. Now let $\mathcal{M}(c) = \{ Dec_k(c) ~|~ k \in \mathcal{K} \}$. Then clearly $|\mathcal{M}(c)| \leq |\mathcal{K}| < |\mathcal{M}|$ so we may choose $m' \notin \mathcal{M}(c)$. By perfect secrecy, we have that $Pr(Enc_k(m') = c) = Pr(Enc_k(m) = c) > 0$. This means that $Dec_k(c) = m'$ for some $k \in \mathcal{K}$. This means $m' \in \mathcal{M}(c)$, a contradiction.\end{proof}

\section{Computational Security}

Proposition~\ref{prop: perfect secrecy implies large keyspace} shows that if we want perfect secrecy, then we need the keyspace $\mathcal{K}$ to be rather large, i.e., at least as large as $\mathcal{M}$. This is impractical computationally, e.g., we don't want to require a 1GB key to encrypt a 1GB file. To allow for a smaller keyspace, we will have to relax the definition of perfect secrecy to only \textit{efficient} adversaries. 

\begin{mydef} A \textit{computational encryption scheme} is a tuple $\Pi = (\operatorname{Gen}, \operatorname{Enc}, \operatorname{Dec})$ such that

\begin{itemize}
	\item For each non-negative integer $n$, we have that $\operatorname{Gen}(n)$ is a random variable with values in a set $\mathcal{K}$ called the keyspace. We assume $\operatorname{Gen}(n)$ runs in polynomial-time in $n$ (it is a probabilistic Turing machine running in polynomial time), that is, it outputs a random key in polynomial time in $n$. We call $n$ the \textit{security parameter}.
	\item The message space and ciphertext space is $\mathcal{M} = \mathcal{C} = \{0,1\}^*$.
	\item For each $k \in \mathcal{K}$ and $m \in \mathcal{M}$, the random variable $\operatorname{Enc}_k(m)$ returns a ciphertext in $\mathcal{C}$ in \textit{polynomial-time}, that is, polynomial in the security parameter $n$ (and the length of $|m|$ ?). In particular, the length of the output is polynomial in $n$. (DOES THE POLYNOMIAL DEPEND ON the key $k$?)
	\item $Dec_k:\mathcal{C} \to \mathcal{M} \cup \{ NULL \}$ is a mapping such that $Dec_k(c) = m$ for all $k, m$ and $c \in \mathcal{C}$ such that $Enc_k(m)$ returns $c$ with positive probability. It is also polynomial in its input length $m$. (It can return $NULL$ if for example $c$ is invalid, not in the output of any encryption).
	\item We sometimes assume that $Enc_k(m)$ is only defined for messages of length $|m| = \ell(n)$ where $\ell(n)$ is a function of $n$ (can't encrypt arbitrarily long messages with a fixed security parameter $n$). If this is the case we call this a \textit{fixed-length private-key encryption scheme for messages of length $\ell(n)$}.
\end{itemize}

\end{mydef}

\begin{eg} The one time pad is a fixed length private-key encryption scheme with of length $\ell(n)=n$. For each security parameter $n$, we have that $Enc_k(m) = m+k \in (\bZ/2\bZ)^n$ is defined for $m \in \{0,1\}^n = (\bZ/2\bZ)^n$. The random variable $Gen(n)$ returns a random key $k \in (\bZ/2\bZ)^n$ in polynomial time, in fact in linear time $O(n)$, as each bit can be randomly chosen in $O(1)$-time.

\end{eg}

We assume that our adversaries are also randomized algorithms running in Polynomial time as follows.

\begin{mydef} An \textit{efficient} adversary to a computational encyrption scheme  $\Pi = (\operatorname{Gen}, \operatorname{Enc}, \operatorname{Dec})$ is an adversary $\mathcal{A}$ as given in Definition~\ref{def: adversary} where for each $m_0,m_1$ of the same length and $c \in \mathcal{C}$ we have that the random variable $\mathcal{A}(m_0,m_1,c)$ returns a random output in $\{0,1\}$ and it runs in polynomial time (it is polynomial in $\max\{|m_0|, |m_1|, |c|\}$).

\end{mydef}

We now modify the experiments $Exp_0$ and $Exp_1$ as follows.

\begin{mydef} Let $\mathcal{A}$ be an efficient adversary to an encryption scheme $\Pi = (\operatorname{Gen}, \operatorname{Enc}, \operatorname{Dec})$. For each fixed security parameter $n$, the adversary chooses two messages $m_0,m_1 \in \mathcal{M}$ of the same length and, in case the scheme is a fixed-length $\ell(n)$, we also require the messages to have length $|m_0| = |m_1| = \ell(n)$. We define two experiments, $\operatorname{Exp}_0$ and $\operatorname{Exp}_1$ as follows. For fixed $b \in \{0,1\}$ we define the random variable $\operatorname{Exp}_b$ by the following experiment:

\begin{enumerate}
	\item Choose $k \in \mathcal{K}$ randomly according to $\operatorname{Gen}(n)$.
	\item Choose $c_b \in \mathcal{C}$ randomly according to $\operatorname{Enc}_k(m_b)$.
	\item Now send $c_b$ to the adversary. They then choose $b' \in \{0,1\}$ randomly according to $\mathcal{A}(m_0, m_1, c_b)$.
	\item Now define $\operatorname{Exp}_b \in \{0,1\}$ to be $1$ if $b = b'$ and $0$ if $b \neq b'$.
\end{enumerate}

 For such fixed $m_0,m_1$ and $n$, we define a random variable $\operatorname{PrivK}_{\mathcal{A}, \Pi}$ as follows:

\begin{enumerate}
	\item Choose $b \in \{0,1\}$ uniformly randomly.
	\item Run $\operatorname{Exp}_b$ as above.
	\item Now define $\operatorname{PrivK}_{\mathcal{A}, \Pi} \in \{0,1\}$ to be the result of $\operatorname{Exp}_b$.

\end{enumerate}

\end{mydef}

Thus the difference now is that there are constraints on what messages $m_0$ and $m_1$ may choose and they depend on the (known) security parameter $n$. This is to ensure that the adversary can't trivially "win" just by looking at the text length of $m_0$, $m_1$ and $c_b$. That is, text-length is not securely hidden by the encryption scheme.

\begin{mydef} A function $f:\bZ_{\geq 0} \to \bR_{>0}$ is said to be negligible if for all $c>0$ there exists $N>0$ such that for all $n>N$ we have $f(n) < n^{-c}$.

\end{mydef}

For instance $2^{-n}$ is negligible.

\begin{mydef} We say that a computational encryption scheme is \textit{EAV-secure} if for all efficient adversaries there exists a negligible function $negl:\bZ_{\geq 0} \to \bR_{>0}$ such that $$\left | Pr(\operatorname{PrivK}_{\mathcal{A}, \Pi} = 1) - \frac{1}{2} \right | < negl(n), $$ where $n$ denotes the security parameter (note that $\operatorname{PrivK}_{\mathcal{A}, \Pi}$ is a random variable for each fixed security parameter $n$).

\end{mydef}

\end{document}